\documentclass[12pt]{article}
\pagestyle{plain}
%\topmargin=-0.5in
%\textheight=9in
%\evensidemargin=-1.5in
%\oddsidemargin=0in
%\setlength{\textwidth}{6.5in}
\usepackage[top=2cm, bottom=2cm, left=1.4cm, right=1.4cm]{geometry}

\usepackage{graphicx}
\usepackage[shortlabels]{enumitem}
\usepackage[hidelinks]{hyperref}
\usepackage{amsthm}
\usepackage{amsmath}
\usepackage{amsfonts}
\usepackage[dvipsnames]{xcolor}
\usepackage{bm}
\usepackage{amssymb}
\usepackage{tikz-cd}
\usepackage{framed}
\usepackage{mdframed}
\usepackage{ulem}

\newtheorem{theorem}{Theorem}
\newtheorem{proposition}{Proposition}
\newtheorem{lemma}{Lemma}
\newtheorem{corollary}{Corollary}

\theoremstyle{definition}
\newtheorem{definition}{Definition}
\newtheorem{example}{Example}

\newcommand{\N}{\mathbb{N}}
\newcommand{\Z}{\mathbb{Z}}
\newcommand{\Q}{\mathbb{Q}}
\newcommand{\R}{\mathbb{R}}
\newcommand{\C}{\mathbb{C}}
\newcommand{\G}{\mathbb{G}}

\newcommand{\sig}{\sigma}
\newcommand{\lam}{\lambda}
\newcommand{\eps}{\varepsilon}

\newcommand{\lp}{\left(}
\newcommand{\rp}{\right)}
\newcommand{\lb}{\left\{}
\newcommand{\rb}{\right\}}
\newcommand{\lab}{\left|}
\newcommand{\rab}{\right|}
\newcommand{\la}{\left\langle}
\newcommand{\ra}{\right\rangle}

\newcommand{\tbf}{\textbf}
\newcommand{\noi}{\noindent}
\newcommand{\incomplete}{\textcolor{red}{INCOMPLETE}}

\newcommand{\inv}{^{-1}}
\newcommand{\sm}{\setminus}
\newcommand{\wt}{\widetilde}
\newcommand{\wh}{\widehat}
\newcommand{\ov}{\overline}
\newcommand{\into}{\hookrightarrow}
\newcommand{\iso}{\cong}

\newcommand{\frakg}{\mathfrak{g}}

\DeclareMathOperator{\Gal}{Gal}
\DeclareMathOperator{\Id}{Id}
\DeclareMathOperator{\SL}{SL}
\DeclareMathOperator{\GL}{GL}
\DeclareMathOperator{\SO}{SO}
\DeclareMathOperator{\SU}{SU}
\DeclareMathOperator{\alg}{alg}

\title{Abstract homomorphisms of non-quasi-split \\
special unitary groups}
\author{Joshua Ruiter}
\date{\today}

\begin{document}
\maketitle
\tableofcontents

\newpage

\section{Generalizing the construction of Kassabov and Sapir}

\subsection{Original construction}

\begin{example}
Let $k$ be a field.  For $1 \le i \neq j \le 3$, let $e_{ij}:k \to \SL_3(k)$ be the map which takes $u \in k$ to the transvection matrix $e_{ij}(u)$. For example,
\[
	e_{13}:k \to \SL_3(k) \qquad e_{13}(u) = \begin{pmatrix} 1 && u \\ & 1 \\ && 1 \end{pmatrix}
\]
Note that $e_{ij}$ is a group homomorphism from $(k,+)$ into $\SL_3(k)$, i.e. $e_{ij}(u) \cdot e_{ij}(v) = e_{ij}(u+v)$. Let $U_{ij} = e_{ij}(k)$, so $e_{ij}$ gives an isomorphism between $(k,+)$ and $U_{ij}$. Let
\begin{align*}
	w_{12} = 
	\begin{pmatrix}
		 & -1 \\
		1 \\
		&& 1
	\end{pmatrix}
	\qquad
	w_{23} = 
	\begin{pmatrix}
		1 \\
		&& 1 \\
		& -1 
	\end{pmatrix}
\end{align*}
and note that
\begin{align}
\label{w relation 1}
	w_{12} \cdot e_{13}(u) \cdot w_{12} \inv &= e_{23}(u) \\
\label{w relation 2}
	w_{23} \cdot e_{13}(u) \cdot w_{23} \inv &= e_{12}(u)
\end{align}
Let $K$ be an algebraically closed field, and let
\[
	\rho:\SL_3(k) \to \GL_m(K)
\]
be a group homomorphism (not necessarily well-behaved with respect to any kind of geometric structure). Let $U = e_{13}(k)$, so $e_{13}$ is an isomorphism between $(k,+)$ and $U$. 

Let $V = \rho(U)$ and let $A = \ov V$ be the Zariski closure of $V$. Since $V$ is a subgroup of $\GL_m(K)$, so is $A$. That is, $A$ is closed under matrix multiplication. We define another binary operation on $A$ as follows:
\begin{align*}
	\mu_\rho:A \times A \to \GL_m(K) \qquad 
	\mu_\rho(a,b) = \Big[ \rho(w_{23}) \cdot a \cdot \rho(w_{23}) \inv, \rho(w_{12}) \cdot b \cdot \rho(w_{12} \inv) \Big]
\end{align*}
A calculation involving the relations (\ref{w relation 1}), (\ref{w relation 2}), and the commutator relation
\[
	\Big[ e_{12}(u),  e_{23}(v) \Big] = e_{13}(uv)
\]
shows that 
\[
	\mu_\rho \Big( \rho \big(e_{13}(u) \big), \rho \big(e_{13}(v) \big) \Big) = e_{13}(uv)
\]
From this, it follows that $\mu_\rho$ maps the subset $V \times V$ (a dense subset of $A \times A$) to $V$. Since $\mu_\rho$ is continuous, it follows that $\mu_\rho$ maps $A \times A$ to $A$.
\end{example}

\newpage

\subsection{Replace $\SL_3$ with arbitrary group}

\begin{definition}
Let $k$ be a field and $K$ be an algebraically closed field. Let $G$ be an algebraic $k$-group, and let
\[
	\rho:G(k) \to \GL_m(K)
\]
be a group homomorphism. Suppose we have functions $e_{12}, e_{13}, e_{23}:k \to G(k)$ such that
\[
	\Big[ e_{12}(u),  e_{23}(v) \Big] = e_{13}(uv)
\]
and suppose have elements $w_{12}, w_{23} \in G(k)$ such that
\begin{align*}
	w_{12} \cdot e_{13}(u) \cdot w_{12} \inv &= e_{23}(u) \\
	w_{23} \cdot e_{13}(u) \cdot w_{23} \inv &= e_{12}(u)
\end{align*}
Let $U = e_{13}(k)$ and let $V = \rho(U)$ and let $A = \ov{V}$. Note that $V$ and $A$ are abelian subgroups of $\GL_m(K)$. Define
\begin{align*}
	\mu_\rho:A \times A \to \GL_m(K) \qquad
	\mu_\rho(a,b) = \Big[ \rho(w_{23}) \cdot a \cdot \rho(w_{23}) \inv, \rho(w_{12}) \cdot b \cdot \rho(w_{12} \inv) \Big]
\end{align*}
Note that $\mu_{\rho}$ is continuous, since it is defined purely in terms of matrix multiplication and inversion of $a, b$, and some fixed elements $\rho(w_{23})$ and $\rho(w_{12})$ in $\GL_m(K)$. 
\end{definition}

\begin{lemma}
\label{multiplicative}
$\mu_\rho \Big( \rho \circ e_{13}(u) , \rho \circ e_{13}(v) \Big) = \rho \circ  e_{13}(uv)$
\end{lemma}
\begin{proof}
\begin{align*}
	\mu_\rho \Big( \rho \circ e_{13}(u) , \rho \circ e_{13}(v) \Big)  &= \Big[ \rho(w_{23}) \cdot \rho (e_{13}(u)) \cdot \rho(w_{23}) \inv, \rho(w_{12}) \cdot \rho(e_{13}(v)) \cdot \rho(w_{12} \inv) \Big] \\
&= \rho \Big[ w_{23} \cdot e_{13}(u) \cdot w_{23} \inv, w_{12} \cdot e_{13}(v) \cdot w_{12} \inv \Big] \\
&= \rho \Big[ e_{12}(u), e_{23}(v) \Big] \\
&=  \rho \circ  e_{13}(uv)
\end{align*}
\end{proof}

\begin{corollary}
$\mu_\rho(V \times V) \subseteq V$, and consequently $\mu_\rho(A \times A) \subseteq A$
\end{corollary}
\begin{proof}
The first inclusion is immediate from Lemma \ref{multiplicative}. The second inclusion follows from this as $\mu_\rho$ is continuous and $V$ is dense in $A$.
\end{proof}

\begin{proposition}
$A$ is a commutative unital algebraic ring under the two binary operations of matrix multiplication and $\mu_\rho$.
\end{proposition}
\begin{proof}
Immediate from Lemma 3.2 from \textit{Linear representations of Chevalley groups over commutative rings}.
\end{proof}

\newpage

\subsection{Replace some notation}

\begin{definition}
Let $k$ be a field and $K$ be an algebraically closed field. Let $G$ be an algebraic $k$-group, and let
\[
	\rho:G(k) \to \GL_m(K)
\]
be a group homomorphism. Suppose we have functions $e_\alpha, e_\beta, e_\gamma:k \to G(k)$ such that
\[
	\Big[ e_\alpha(u),  e_\beta(v) \Big] = e_\gamma(uv)
\]
and suppose have elements $w_{\gamma \to \beta}, w_{\gamma \to \alpha} \in G(k)$ such that
\begin{align*}
	w_{\gamma \to \beta} \cdot e_\gamma (u) \cdot w_{\gamma \to \beta} \inv &= e_\beta (u) \\
	w_{\gamma \to \alpha} \cdot e_\gamma (u) \cdot w_{\gamma \to \alpha} \inv &= e_\alpha (u)
\end{align*}
Let $U = e_\gamma (k)$ and let $V = \rho(U)$ and let $A = \ov{V}$. Note that $V$ and $A$ are abelian subgroups of $\GL_m(K)$. Define
\begin{align*}
	\mu_\rho:A \times A \to \GL_m(K) \qquad
	\mu_\rho(a,b) = \Big[ \rho(w_{\gamma \to \alpha}) \cdot a \cdot \rho(w_{\gamma \to \alpha}) \inv, \rho(w_{\gamma \to \beta}) \cdot b \cdot \rho(w_{\gamma \to \beta} \inv) \Big]
\end{align*}
Note that $\mu_{\rho}$ is continuous, since it is defined purely in terms of matrix multiplication and inversion of $a, b$, and some fixed elements $\rho(w_{\gamma \to \alpha})$ and $\rho(w_{\gamma \to \beta})$ in $\GL_m(K)$. 
\end{definition}

\begin{lemma}
\label{multiplicative}
$\mu_\rho \Big( \rho  \circ e_\gamma(u), \rho \circ e_\gamma(v) \Big) = \rho \circ e_\gamma(uv)$
\end{lemma}
\begin{proof}
\begin{align*}
	\mu_\rho \Big( \rho  \circ e_\gamma(u), \rho \circ e_\gamma(v) \Big) &= \Big[ \rho(w_{\gamma \to \alpha}) \cdot \rho \circ e_\gamma(u) \cdot \rho(w_{\gamma \to \alpha}) \inv, \rho(w_{\gamma \to \beta}) \cdot \rho \circ e_\gamma(v) \cdot \rho(w_{\gamma \to \beta} \inv) \Big] \\
&= \rho \Big[ w_{\gamma \to \alpha} \cdot e_\gamma(u) \cdot w_{\gamma \to \alpha} \inv, w_{\gamma \to \beta} \cdot e_\gamma(v) \cdot w_{\gamma \to \beta} \inv \Big] \\
&= \rho \Big[ e_\alpha(u), e_\beta(v) \Big] \\
&=  \rho \circ e_\gamma(uv)
\end{align*}
\end{proof}

\begin{corollary}
$\mu_\rho(V \times V) \subseteq V$, and consequently $\mu_\rho(A \times A) \subseteq A$
\end{corollary}
\begin{proof}
The first inclusion is immediate from Lemma \ref{multiplicative}. The second inclusion follows from this as $\mu_\rho$ is continuous and $V$ is dense in $A$.
\end{proof}

\begin{proposition}
$A$ is a commutative unital algebraic ring under the two binary operations of matrix multiplication and $\mu_\rho$.
\end{proposition}
\begin{proof}
Immediate from Lemma 3.2 from \textit{Linear representations of Chevalley groups over commutative rings}.
\end{proof}

\begin{example}
Let $G = \SL_n$ and let $e_\alpha = e_{12}$ and $e_\beta = e_{23}$ and $e_\gamma = e_{13}$. Then we have
\[
	[e_\alpha(u), e_\beta(v)] = e_\gamma(uv)
\]
Then with 
\begin{align*}
	w_{\gamma \to \beta} &= w_{12} \\
	w_{\gamma \to \alpha} &= w_{23}
\end{align*}
we have
\begin{align*}
	w_{\gamma \to \beta} \cdot e_\gamma(u) \cdot w_{\gamma \to \beta} \inv &= e_\beta(u) \\
	w_{\gamma \to \alpha} \cdot e_\gamma(u) \cdot w_{\gamma \to \alpha} \inv &= e_\alpha(u)
\end{align*}
\begin{enumerate}[(a)]
\item Suppose $k=K = K^{\alg}$ and $\rho$ is the inclusion $\SL_n(k) \into \GL_n(k)$. Then $U = V = e_{13}(k)$, and $V$ is already closed so $A = V = U$ (technically $U$ is a subset of a different group than $V = A$, but whatever). As an abelian group (under the operation of matrix multiplication), $A$ is isomorphic to $(k, +)$. Since $\rho$ is just inclusion, Lemma \ref{multiplicative} says that
\[
	\mu_\rho \Big( e_\gamma(u), e_\gamma(v) \Big) = e_\gamma(uv)
\]
i.e. $e_\gamma$ is a ring isomorphism $(k, +, \cdot) \to (A, \cdot, \mu_\rho)$.

\item Suppose $k = K = \C$ and $\rho:\SL_n(\C) \to \GL_n(\C)$ is the composition of inclusion with entrywise complex conjugation. As in the previous example, $A = V = U$. Also as above, $A$ is isomorphic to $(k,+)$ as an abelian group under matrix multiplication. Now Lemma \ref{multiplicative} tells us
\[
	\mu_\rho \Big(  e_\gamma( \ov u), e_\gamma( \ov v) \Big) = e_\gamma(\ov{uv})
\]
i.e. the map
\[
	f:\C \to A \qquad f(u) = e_\gamma( \ov u)
\]
is a ring isomorphism.
%\begin{align*}
%	f(u+v) &= e_\gamma( \ov{u+v} ) = e_\gamma( \ov u + \ov v) = e_\gamma( \ov u) \cdot e_\gamma(\ov v) = f(u) \cdot f(v) \\
%	f(uv) &= e_\gamma( \ov{uv}) = \mu_\rho \Big( e_\gamma(\ov u), e_\gamma( \ov v) \Big) = \mu_\rho \Big( f(u), f(v) \Big) 
%\end{align*}

\item Let $L/k$ be a Galois extension, and let $K = L^{\alg} = k^{\alg}$ be the algebraic closure. Fix an embedding $L \into K$. Let $\sig \in \Gal(L/k)$, and let $\rho:\SL_n(L) \to \GL_n(K)$ be the composition of inclusion and entrywise application of $\sig$. Then again $A = V = U$, though technically $U = e_{13}(L) \subseteq \SL_n(L)$, and $A = V = e_{13}(L) \subseteq \GL_n(K)$. As abelian groups, $(L, +) \iso (A, \cdot)$. Lemma \ref{multiplicative} says
\[
	\mu_\rho \Big( e_\gamma( \sig u), e_\gamma ( \sig v) \Big) = e_\gamma \Big( \sig(uv) \Big)
\]
so the map
\[
	f:L \to A \qquad f(u) = e_\gamma( \sig u)
\]
gives a ring isomorphism $(L, +, \cdot) \iso (A, \cdot, \mu_\rho)$.

\end{enumerate}
\end{example}

\newpage

\subsection{Generalize commutator relation}

\begin{definition}
Let $k$ be a field and $K$ be an algebraically closed field. Let $G$ be an algebraic $k$-group, and let
\[
	\rho:G(k) \to \GL_m(K)
\]
be a group homomorphism. Suppose we have functions $e_\alpha, e_\beta, e_\gamma:k \to G(k)$ such that
\[
	\Big[ e_\alpha(u),  e_\beta(v) \Big] = e_\gamma \Big( N_{\alpha \beta}(u,v) \Big)
\]
and suppose we have an element $h_\gamma \in G(k)$ such that
\[
	h_\gamma \cdot  e_\gamma \Big( N_{\alpha \beta}(u,v) \Big) \cdot h_\gamma \inv = uv
\]
and suppose we have elements $w_{\gamma \to \beta}, w_{\gamma \to \alpha} \in G(k)$ such that
\begin{align*}
	w_{\gamma \to \beta} \cdot e_\gamma (u) \cdot w_{\gamma \to \beta} \inv &= e_\beta (u) \\
	w_{\gamma \to \alpha} \cdot e_\gamma (u) \cdot w_{\gamma \to \alpha} \inv &= e_\alpha (u)
\end{align*}
Let $U = e_\gamma (k)$ and let $V = \rho(U)$ and let $A = \ov{V}$. Note that $V$ and $A$ are abelian subgroups of $\GL_m(K)$. Define
\begin{align*}
	\mu_\rho:A \times A \to \GL_m(K) \qquad
	\mu_\rho(a,b) = \rho(h_\gamma) \cdot \Big[ \rho(w_{\gamma \to \alpha}) \cdot a \cdot \rho(w_{\gamma \to \alpha}) \inv, \rho(w_{\gamma \to \beta}) \cdot b \cdot \rho(w_{\gamma \to \beta} \inv) \Big] \cdot \rho(h_\gamma) \inv
\end{align*}
Note that $\mu_{\rho}$ is continuous, since it is defined purely in terms of matrix multiplication and inversion of $a, b$, and some fixed elements $\rho(w_{\gamma \to \alpha})$ and $\rho(w_{\gamma \to \beta})$ in $\GL_m(K)$. 
\end{definition}

\begin{lemma}
\label{multiplicative}
$\mu_\rho \Big( \rho  \circ e_\gamma(u), \rho \circ e_\gamma(v) \Big) = \rho \circ e_\gamma(uv)$
\end{lemma}
\begin{proof}
\begin{align*}
	\mu_\rho \Big( \rho  \circ e_\gamma(u), \rho \circ e_\gamma(v) \Big) &= \rho(h_\gamma) \cdot \Big[ \rho(w_{\gamma \to \alpha}) \cdot \rho  \circ e_\gamma(u) \cdot \rho(w_{\gamma \to \alpha}) \inv, \rho(w_{\gamma \to \beta}) \cdot \rho \circ e_\gamma(v) \cdot \rho(w_{\gamma \to \beta} \inv) \Big] \cdot \rho(h_\gamma) \inv \\
&= \rho \Big( h_\gamma \cdot \Big[ w_{\gamma \to \alpha} \cdot e_\gamma(u) \cdot w_{\gamma \to \alpha} \inv, w_{\gamma \to \beta} \cdot e_\gamma(v) \cdot w_{\gamma \to \beta} \inv \Big] \cdot h_\gamma \inv \Big) \\
&= \rho \Big( h_\gamma \cdot \big[ e_\alpha(u), e_\beta(v) \big] \cdot h_\gamma \inv \Big) \\
&= \rho \Big( h_\gamma \cdot e_\gamma \big( N_{\alpha \beta}(u,v) \big) \cdot h_\gamma \inv \Big) \\
&= \rho \Big( e_\gamma (uv) \Big) \\
&= \rho \circ e_\gamma(uv)
\end{align*}
\end{proof}

\begin{corollary}
$\mu_\rho(V \times V) \subseteq V$, and consequently $\mu_\rho(A \times A) \subseteq A$
\end{corollary}
\begin{proof}
The first inclusion is immediate from Lemma \ref{multiplicative}. The second inclusion follows from this as $\mu_\rho$ is continuous and $V$ is dense in $A$.
\end{proof}

\begin{proposition}
$A$ is a commutative unital algebraic ring under the two binary operations of matrix multiplication and $\mu_\rho$.
\end{proposition}
\begin{proof}
Immediate from Lemma 3.2 from \textit{Linear representations of Chevalley groups over commutative rings}.
\end{proof}

\end{document}