\documentclass[12pt]{article}
\pagestyle{plain}
%\topmargin=-0.5in
%\textheight=9in
%\evensidemargin=-1.5in
%\oddsidemargin=0in
%\setlength{\textwidth}{6.5in}
\usepackage[top=2cm, bottom=2cm, left=1.4cm, right=1.4cm]{geometry}

\usepackage{graphicx}
\usepackage[shortlabels]{enumitem}
\usepackage[hidelinks]{hyperref}
\usepackage{amsthm}
\usepackage{amsmath}
\usepackage{amsfonts}
\usepackage[dvipsnames]{xcolor}
\usepackage{bm}
\usepackage{amssymb}
\usepackage{tikz-cd}
\usepackage{framed}
\usepackage{mdframed}
\usepackage{ulem}

\newtheorem{theorem}{Theorem}
\newtheorem{proposition}{Proposition}
\newtheorem{lemma}{Lemma}
\newtheorem{corollary}{Corollary}

\theoremstyle{definition}
\newtheorem{definition}{Definition}
\newtheorem{example}{Example}

\newcommand{\N}{\mathbb{N}}
\newcommand{\Z}{\mathbb{Z}}
\newcommand{\Q}{\mathbb{Q}}
\newcommand{\R}{\mathbb{R}}
\newcommand{\C}{\mathbb{C}}
\newcommand{\G}{\mathbb{G}}

\newcommand{\sig}{\sigma}
\newcommand{\lam}{\lambda}
\newcommand{\eps}{\varepsilon}

\newcommand{\lp}{\left(}
\newcommand{\rp}{\right)}
\newcommand{\lb}{\left\{}
\newcommand{\rb}{\right\}}
\newcommand{\lab}{\left|}
\newcommand{\rab}{\right|}
\newcommand{\la}{\left\langle}
\newcommand{\ra}{\right\rangle}

\newcommand{\tbf}{\textbf}
\newcommand{\noi}{\noindent}
\newcommand{\incomplete}{\textcolor{red}{INCOMPLETE}}

\newcommand{\inv}{^{-1}}
\newcommand{\sm}{\setminus}
\newcommand{\wt}{\widetilde}
\newcommand{\wh}{\widehat}
\newcommand{\ov}{\overline}
\newcommand{\into}{\hookrightarrow}
\newcommand{\iso}{\cong}

\newcommand{\frakg}{\mathfrak{g}}

\DeclareMathOperator{\Gal}{Gal}
\DeclareMathOperator{\Id}{Id}
\DeclareMathOperator{\SL}{SL}
\DeclareMathOperator{\GL}{GL}
\DeclareMathOperator{\SO}{SO}
\DeclareMathOperator{\SU}{SU}
\DeclareMathOperator{\alg}{alg}

\title{Abstract homomorphisms of non-quasi-split \\
special unitary groups}
\author{Joshua Ruiter}
\date{\today}

\begin{document}
\maketitle
\tableofcontents

\newpage

\section{Generalizing the construction of Kassabov and Sapir}

\subsection{Original construction}

\begin{example}
Let $k$ be a field.  For $1 \le i \neq j \le 3$, let $e_{ij}:k \to \SL_3(k)$ be the map which takes $u \in k$ to the transvection matrix $e_{ij}(u)$. For example,
\[
	e_{13}:k \to \SL_3(k) \qquad e_{13}(u) = \begin{pmatrix} 1 && u \\ & 1 \\ && 1 \end{pmatrix}
\]
Note that $e_{ij}$ is a group homomorphism from $(k,+)$ into $\SL_3(k)$, i.e. $e_{ij}(u) \cdot e_{ij}(v) = e_{ij}(u+v)$. Let $U_{ij} = e_{ij}(k)$, so $e_{ij}$ gives an isomorphism between $(k,+)$ and $U_{ij}$. Let
\begin{align*}
	w_{12} = 
	\begin{pmatrix}
		 & -1 \\
		1 \\
		&& 1
	\end{pmatrix}
	\qquad
	w_{23} = 
	\begin{pmatrix}
		1 \\
		&& 1 \\
		& -1 
	\end{pmatrix}
\end{align*}
and note that
\begin{align}
\label{w relation 1}
	w_{12} \cdot e_{13}(u) \cdot w_{12} \inv &= e_{23}(u) \\
\label{w relation 2}
	w_{23} \cdot e_{13}(u) \cdot w_{23} \inv &= e_{12}(u)
\end{align}
Let $K$ be an algebraically closed field, and let
\[
	\rho:\SL_3(k) \to \GL_m(K)
\]
be a group homomorphism (not necessarily well-behaved with respect to any kind of geometric structure). Let $U = e_{13}(k)$, so $e_{13}$ is an isomorphism between $(k,+)$ and $U$. 

Let $V = \rho(U)$ and let $A = \ov V$ be the Zariski closure of $V$. Since $V$ is a subgroup of $\GL_m(K)$, so is $A$. That is, $A$ is closed under matrix multiplication. We define another binary operation on $A$ as follows:
\begin{align*}
	\mu_\rho:A \times A \to \GL_m(K) \qquad 
	\mu_\rho(a,b) = \Big[ \rho(w_{23}) \cdot a \cdot \rho(w_{23}) \inv, \rho(w_{12}) \cdot b \cdot \rho(w_{12} \inv) \Big]
\end{align*}
A calculation involving the relations (\ref{w relation 1}), (\ref{w relation 2}), and the commutator relation
\[
	\Big[ e_{12}(u),  e_{23}(v) \Big] = e_{13}(uv)
\]
shows that 
\[
	\mu_\rho \Big( \rho \big(e_{13}(u) \big), \rho \big(e_{13}(v) \big) \Big) = e_{13}(uv)
\]
From this, it follows that $\mu_\rho$ maps the subset $V \times V$ (a dense subset of $A \times A$) to $V$. Since $\mu_\rho$ is continuous, it follows that $\mu_\rho$ maps $A \times A$ to $A$.
\end{example}

\newpage

\subsection{Replace $\SL_3$ with arbitrary group}

\begin{definition}
Let $k$ be a field and $K$ be an algebraically closed field. Let $G$ be an algebraic $k$-group, and let
\[
	\rho:G(k) \to \GL_m(K)
\]
be a group homomorphism. Suppose we have functions $e_{12}, e_{13}, e_{23}:k \to G(k)$ such that
\[
	\Big[ e_{12}(u),  e_{23}(v) \Big] = e_{13}(uv)
\]
and suppose have elements $w_{12}, w_{23} \in G(k)$ such that
\begin{align*}
	w_{12} \cdot e_{13}(u) \cdot w_{12} \inv &= e_{23}(u) \\
	w_{23} \cdot e_{13}(u) \cdot w_{23} \inv &= e_{12}(u)
\end{align*}
Let $U = e_{13}(k)$ and let $V = \rho(U)$ and let $A = \ov{V}$. Note that $V$ and $A$ are abelian subgroups of $\GL_m(K)$. Define
\begin{align*}
	\mu_\rho:A \times A \to \GL_m(K) \qquad
	\mu_\rho(a,b) = \Big[ \rho(w_{23}) \cdot a \cdot \rho(w_{23}) \inv, \rho(w_{12}) \cdot b \cdot \rho(w_{12} \inv) \Big]
\end{align*}
Note that $\mu_{\rho}$ is continuous, since it is defined purely in terms of matrix multiplication and inversion of $a, b$, and some fixed elements $\rho(w_{23})$ and $\rho(w_{12})$ in $\GL_m(K)$. 
\end{definition}

\begin{lemma}
\label{multiplicative}
$\mu_\rho \Big( \rho \circ e_{13}(u) , \rho \circ e_{13}(v) \Big) = \rho \circ  e_{13}(uv)$
\end{lemma}
\begin{proof}
\begin{align*}
	\mu_\rho \Big( \rho \circ e_{13}(u) , \rho \circ e_{13}(v) \Big)  &= \Big[ \rho(w_{23}) \cdot \rho (e_{13}(u)) \cdot \rho(w_{23}) \inv, \rho(w_{12}) \cdot \rho(e_{13}(v)) \cdot \rho(w_{12} \inv) \Big] \\
&= \rho \Big[ w_{23} \cdot e_{13}(u) \cdot w_{23} \inv, w_{12} \cdot e_{13}(v) \cdot w_{12} \inv \Big] \\
&= \rho \Big[ e_{12}(u), e_{23}(v) \Big] \\
&=  \rho \circ  e_{13}(uv)
\end{align*}
\end{proof}

\begin{corollary}
$\mu_\rho(V \times V) \subseteq V$, and consequently $\mu_\rho(A \times A) \subseteq A$
\end{corollary}
\begin{proof}
The first inclusion is immediate from Lemma \ref{multiplicative}. The second inclusion follows from this as $\mu_\rho$ is continuous and $V$ is dense in $A$.
\end{proof}

\begin{proposition}
$A$ is a commutative unital algebraic ring under the two binary operations of matrix multiplication and $\mu_\rho$.
\end{proposition}
\begin{proof}
Immediate from Lemma 3.2 from \textit{Linear representations of Chevalley groups over commutative rings}.
\end{proof}

\newpage

\subsection{Replace some notation}

\begin{definition}
Let $k$ be a field and $K$ be an algebraically closed field. Let $G$ be an algebraic $k$-group, and let
\[
	\rho:G(k) \to \GL_m(K)
\]
be a group homomorphism. Suppose we have functions $e_\alpha, e_\beta, e_\gamma:k \to G(k)$ such that
\[
	\Big[ e_\alpha(u),  e_\beta(v) \Big] = e_\gamma(uv)
\]
and suppose have elements $w_{\gamma \to \beta}, w_{\gamma \to \alpha} \in G(k)$ such that
\begin{align*}
	w_{\gamma \to \beta} \cdot e_\gamma (u) \cdot w_{\gamma \to \beta} \inv &= e_\beta (u) \\
	w_{\gamma \to \alpha} \cdot e_\gamma (u) \cdot w_{\gamma \to \alpha} \inv &= e_\alpha (u)
\end{align*}
Let $U = e_\gamma (k)$ and let $V = \rho(U)$ and let $A = \ov{V}$. Note that $V$ and $A$ are abelian subgroups of $\GL_m(K)$. Define
\begin{align*}
	\mu_\rho:A \times A \to \GL_m(K) \qquad
	\mu_\rho(a,b) = \Big[ \rho(w_{\gamma \to \alpha}) \cdot a \cdot \rho(w_{\gamma \to \alpha}) \inv, \rho(w_{\gamma \to \beta}) \cdot b \cdot \rho(w_{\gamma \to \beta} \inv) \Big]
\end{align*}
Note that $\mu_{\rho}$ is continuous, since it is defined purely in terms of matrix multiplication and inversion of $a, b$, and some fixed elements $\rho(w_{\gamma \to \alpha})$ and $\rho(w_{\gamma \to \beta})$ in $\GL_m(K)$. 
\end{definition}

\begin{lemma}
\label{multiplicative}
$\mu_\rho \Big( \rho  \circ e_\gamma(u), \rho \circ e_\gamma(v) \Big) = \rho \circ e_\gamma(uv)$
\end{lemma}
\begin{proof}
\begin{align*}
	\mu_\rho \Big( \rho  \circ e_\gamma(u), \rho \circ e_\gamma(v) \Big) &= \Big[ \rho(w_{\gamma \to \alpha}) \cdot \rho \circ e_\gamma(u) \cdot \rho(w_{\gamma \to \alpha}) \inv, \rho(w_{\gamma \to \beta}) \cdot \rho \circ e_\gamma(v) \cdot \rho(w_{\gamma \to \beta} \inv) \Big] \\
&= \rho \Big[ w_{\gamma \to \alpha} \cdot e_\gamma(u) \cdot w_{\gamma \to \alpha} \inv, w_{\gamma \to \beta} \cdot e_\gamma(v) \cdot w_{\gamma \to \beta} \inv \Big] \\
&= \rho \Big[ e_\alpha(u), e_\beta(v) \Big] \\
&=  \rho \circ e_\gamma(uv)
\end{align*}
\end{proof}

\begin{corollary}
$\mu_\rho(V \times V) \subseteq V$, and consequently $\mu_\rho(A \times A) \subseteq A$
\end{corollary}
\begin{proof}
The first inclusion is immediate from Lemma \ref{multiplicative}. The second inclusion follows from this as $\mu_\rho$ is continuous and $V$ is dense in $A$.
\end{proof}

\begin{proposition}
$A$ is a commutative unital algebraic ring under the two binary operations of matrix multiplication and $\mu_\rho$.
\end{proposition}
\begin{proof}
Immediate from Lemma 3.2 from \textit{Linear representations of Chevalley groups over commutative rings}.
\end{proof}

\begin{example}
Let $G = \SL_n$ and let $e_\alpha = e_{12}$ and $e_\beta = e_{23}$ and $e_\gamma = e_{13}$. Then we have
\[
	[e_\alpha(u), e_\beta(v)] = e_\gamma(uv)
\]
Then with 
\begin{align*}
	w_{\gamma \to \beta} &= w_{12} \\
	w_{\gamma \to \alpha} &= w_{23}
\end{align*}
we have
\begin{align*}
	w_{\gamma \to \beta} \cdot e_\gamma(u) \cdot w_{\gamma \to \beta} \inv &= e_\beta(u) \\
	w_{\gamma \to \alpha} \cdot e_\gamma(u) \cdot w_{\gamma \to \alpha} \inv &= e_\alpha(u)
\end{align*}
\begin{enumerate}[(a)]
\item Suppose $k=K = K^{\alg}$ and $\rho$ is the inclusion $\SL_n(k) \into \GL_n(k)$. Then $U = V = e_{13}(k)$, and $V$ is already closed so $A = V = U$ (technically $U$ is a subset of a different group than $V = A$, but whatever). As an abelian group (under the operation of matrix multiplication), $A$ is isomorphic to $(k, +)$. Since $\rho$ is just inclusion, Lemma \ref{multiplicative} says that
\[
	\mu_\rho \Big( e_\gamma(u), e_\gamma(v) \Big) = e_\gamma(uv)
\]
i.e. $e_\gamma$ is a ring isomorphism $(k, +, \cdot) \to (A, \cdot, \mu_\rho)$.

\item Suppose $k = K = \C$ and $\rho:\SL_n(\C) \to \GL_n(\C)$ is the composition of inclusion with entrywise complex conjugation. As in the previous example, $A = V = U$. Also as above, $A$ is isomorphic to $(k,+)$ as an abelian group under matrix multiplication. Now Lemma \ref{multiplicative} tells us
\[
	\mu_\rho \Big(  e_\gamma( \ov u), e_\gamma( \ov v) \Big) = e_\gamma(\ov{uv})
\]
i.e. the map
\[
	f:\C \to A \qquad f(u) = e_\gamma( \ov u)
\]
is a ring isomorphism.
%\begin{align*}
%	f(u+v) &= e_\gamma( \ov{u+v} ) = e_\gamma( \ov u + \ov v) = e_\gamma( \ov u) \cdot e_\gamma(\ov v) = f(u) \cdot f(v) \\
%	f(uv) &= e_\gamma( \ov{uv}) = \mu_\rho \Big( e_\gamma(\ov u), e_\gamma( \ov v) \Big) = \mu_\rho \Big( f(u), f(v) \Big) 
%\end{align*}

\item Let $k$ be any field, let $k^{\alg}$ be the algebraic closure, and fix an embedding $k \into k^{\alg}.$ Let $\sig:k \to k$ be a field automorphism. Let $\rho:\SL_n(k) \to \GL_n(K)$ be the composition of inclusion and entrywise application of $\sig$. Then again $A = V = U$, though technically $U = e_{13}(L) \subseteq \SL_n(k)$, and $A = V = e_{13}(k) \subseteq \GL_n(K)$. As abelian groups, $(k, +) \iso (A, \cdot)$. Lemma \ref{multiplicative} says
\[
	\mu_\rho \Big( e_\gamma( \sig u), e_\gamma ( \sig v) \Big) = e_\gamma \Big( \sig(uv) \Big)
\]
so the map
\[
	f:L \to A \qquad f(u) = e_\gamma( \sig u)
\]
gives a ring isomorphism $(k, +, \cdot) \iso (A, \cdot, \mu_\rho)$.

\item Let $k$ be any field, let $k^{\alg}$ be the algebraic closure, and fix an embedding $k \into k^{\alg}$. Let $\tau:k^{\alg} \to k^{\alg}$ be a field automorphism. Let $\rho:\SL_n(k) \to \GL_n(K)$ be the composition of inclusion and entrywise application of $\sig$. Then
\begin{align*}
	U &= e_\gamma(k) \subseteq \SL_n(k) \\
	V &= e_\gamma(\tau k) \subseteq \GL_n(K) \\
	A &= V
\end{align*}
As above, $(k, +, \cdot) \iso (A, \cdot \mu_\rho)$ via the map $k \to A, u \mapsto e_\gamma(\tau u)$.
\end{enumerate}
\end{example}

\begin{example}
Let $G = \SL_n$ and let $T \subset G$ be the diagonal torus. Let $\Phi = \Phi(G, T)$ be the associated root system, of type $A_{n-1}$. Concretely,
\begin{align*}
	\Phi &= \lb \alpha_{ij} : 1 \le i \neq j \le n \rb 
\end{align*}
where $\alpha_{ij} = \alpha_i - \alpha_j$ and $\alpha_i:T \to k^\times$ is the character that picks off the $i$th diagonal entry. Let $\alpha, \beta \in \Phi$ such that $\alpha+\beta = \gamma \in \Phi$. That is, $\alpha = \alpha_{ij}$ and $\beta = \alpha_{j\ell}$ and $\gamma = \alpha_{i\ell}$ for some three distinct $i,j,\ell$. There are maps $e_\alpha, e_\beta, e_\gamma:k \to \SL_n(k)$ such that, for example,
\[
	e_\alpha(k) = \lb x \in \SL_n(k) : txt \inv = \alpha(t) x, \forall t \in T(k) \rb
\]
Because $\alpha + \beta = \gamma$, we have a commutator relation
\[
	\Big[ e_\alpha(u), e_\beta(v) \Big] = e_\gamma( \pm uv )
\]
If the sign is negative, just reverse the roles of $\alpha$ and $\beta$ so that the sign is positive, so we may assume that
\[
	\Big[ e_\alpha(u), e_\beta(v) \Big] = e_\gamma( uv )
\]
Also, there are ``Weyl group elements" $w_{\gamma \to \alpha}, w_{\gamma \to \beta} \in \SL_n(k)$ so that
\begin{align*}
	w_{\gamma \to \beta} \cdot e_\gamma(u) \cdot w_{\gamma \to \beta} \inv &= e_\beta(u) \\
	w_{\gamma \to \alpha} \cdot e_\gamma(u) \cdot w_{\gamma \to \alpha} \inv &= e_\alpha(u)
\end{align*}
I put ``Weyl group elements" in quotation marks because techincally the Weyl group is the quotient $N_G(T)/ Z_G(T) = N_G(T)/T$, and so elements of the Weyl group are not in $G(k)$. The elements above are, literally speaking, elements of the normalizer $N_G(T)$, representing elements of the quotient which is the Weyl group.
\end{example}

\newpage

\subsection{Generalize commutator relation}

\begin{definition}
Let $k$ be a field and $K$ be an algebraically closed field. Let $G$ be an algebraic $k$-group, and let
\[
	\rho:G(k) \to \GL_m(K)
\]
be a group homomorphism. Suppose we have functions $e_\alpha, e_\beta, e_\gamma:k \to G(k)$ such that
\[
	\Big[ e_\alpha(u),  e_\beta(v) \Big] = e_\gamma \Big( N_{\alpha \beta}(u,v) \Big)
\]
and suppose we have an element $h_\gamma \in G(k)$ such that
\[
	h_\gamma \cdot  e_\gamma \Big( N_{\alpha \beta}(u,v) \Big) \cdot h_\gamma \inv = e_\gamma( uv )
\]
and suppose we have elements $w_{\gamma \to \beta}, w_{\gamma \to \alpha} \in G(k)$ such that
\begin{align*}
	w_{\gamma \to \beta} \cdot e_\gamma (u) \cdot w_{\gamma \to \beta} \inv &= e_\beta (u) \\
	w_{\gamma \to \alpha} \cdot e_\gamma (u) \cdot w_{\gamma \to \alpha} \inv &= e_\alpha (u)
\end{align*}
Let $U = e_\gamma (k)$ and let $V = \rho(U)$ and let $A = \ov{V}$. Note that $V$ and $A$ are abelian subgroups of $\GL_m(K)$. Define
\begin{align*}
	\mu_\rho:A \times A \to \GL_m(K) \qquad
	\mu_\rho(a,b) = \rho(h_\gamma) \cdot \Big[ \rho(w_{\gamma \to \alpha}) \cdot a \cdot \rho(w_{\gamma \to \alpha}) \inv, \rho(w_{\gamma \to \beta}) \cdot b \cdot \rho(w_{\gamma \to \beta} \inv) \Big] \cdot \rho(h_\gamma) \inv
\end{align*}
Note that $\mu_{\rho}$ is continuous, since it is defined purely in terms of matrix multiplication and inversion of $a, b$, and some fixed elements $\rho(w_{\gamma \to \alpha})$ and $\rho(w_{\gamma \to \beta})$ in $\GL_m(K)$. 
\end{definition}

\begin{lemma}
\label{multiplicative}
$\mu_\rho \Big( \rho  \circ e_\gamma(u), \rho \circ e_\gamma(v) \Big) = \rho \circ e_\gamma(uv)$
\end{lemma}
\begin{proof}
\begin{align*}
	\mu_\rho \Big( \rho  \circ e_\gamma(u), \rho \circ e_\gamma(v) \Big) &= \rho(h_\gamma) \cdot \Big[ \rho(w_{\gamma \to \alpha}) \cdot \rho  \circ e_\gamma(u) \cdot \rho(w_{\gamma \to \alpha}) \inv, \rho(w_{\gamma \to \beta}) \cdot \rho \circ e_\gamma(v) \cdot \rho(w_{\gamma \to \beta} \inv) \Big] \cdot \rho(h_\gamma) \inv \\
&= \rho \Big( h_\gamma \cdot \Big[ w_{\gamma \to \alpha} \cdot e_\gamma(u) \cdot w_{\gamma \to \alpha} \inv, w_{\gamma \to \beta} \cdot e_\gamma(v) \cdot w_{\gamma \to \beta} \inv \Big] \cdot h_\gamma \inv \Big) \\
&= \rho \Big( h_\gamma \cdot \big[ e_\alpha(u), e_\beta(v) \big] \cdot h_\gamma \inv \Big) \\
&= \rho \Big( h_\gamma \cdot e_\gamma \big( N_{\alpha \beta}(u,v) \big) \cdot h_\gamma \inv \Big) \\
&= \rho \Big( e_\gamma (uv) \Big) \\
&= \rho \circ e_\gamma(uv)
\end{align*}
\end{proof}

\begin{corollary}
$\mu_\rho(V \times V) \subseteq V$, and consequently $\mu_\rho(A \times A) \subseteq A$
\end{corollary}
\begin{proof}
The first inclusion is immediate from Lemma \ref{multiplicative}. The second inclusion follows from this as $\mu_\rho$ is continuous and $V$ is dense in $A$.
\end{proof}

\begin{proposition}
$A$ is a commutative unital algebraic ring under the two binary operations of matrix multiplication and $\mu_\rho$.
\end{proposition}
\begin{proof}
Immediate from Lemma 3.2 from \textit{Linear representations of Chevalley groups over commutative rings}.
\end{proof}

\begin{example}
Split special orthogonal group, where $h$ is needed \incomplete
\end{example}

\begin{example}
Quasi-split special unitary group, where $h$ is needed \incomplete
\end{example}

\noi In the following definition, an algebraic $k$-group is a functor from the category of $k$-algebras to the category of groups. So if $G$ is an algebraic $k$-group, then $G$ is not literally a group, but $G(k)$ is a group, called the group of $k$-rational points. A morphism of algebraic groups is a natural transformation $G \to H$, which comes with literal group homomorphisms $G(R) \to H(R)$ for every $k$-algebra $R$.

\begin{definition}
\label{isotropic group}
Let $G$ be an isotropic\footnote{Isotropic just means it contains a torus of some positive dimension.} reductive algebraic $k$-group, and let $\frakg$ be the Lie algebra of $G$. Let $S \subseteq G$ be a maximal\footnote{$S$ is maximal among $k$-split tori, not necessarily maximal among all tori} $k$-split torus. A \tbf{character} of $S$ is a morphism of algebraic groups $S \to \G_m$. The set of characters is denoted $X(S)$. For $\alpha \in X(S)$, let
\[
	\frakg_\alpha(k) = \lb X \in \frakg(k) : s \cdot X = \alpha(s) X, \forall s \in S(k) \rb
\]
For every $\alpha \in X(S)$, $\frakg_\alpha(k)$ is a Lie subalgebra of $\frakg(k)$. If $k$ is algebraically closed and characteristic zero, then $\dim(\frakg_\alpha(k))$ is always zero or one, but for an arbitrary field $k$ it may have larger dimension. Let
\begin{align*}
	d_\alpha &= \dim_k(\frakg_\alpha(k)) \\
	V_\alpha &= (\G_a)^{d_\alpha}
\end{align*}
We may think of $V_\alpha(k)$ as a $k$-vector space with dimension $d_\alpha$. In fact, as a vector space, $V_\alpha(k) \iso \frakg_\alpha(k)$, though we do not think of $V_\alpha(k)$ as being a subset of the Lie algebra $\frakg(k)$ or having any kind of Lie algebra structure. When $k$ is algebraically closed and characteristic zero, $V_\alpha$ is just $\lb 0 \rb$ or $k$, but in general it may have larger dimension. 

 The \tbf{relative root system of $G$ with respect to $S$} is
\[
	\Phi = \Phi(G,S) = \lb \alpha \in X(S) : \frakg_\alpha(k) \neq 0 \rb = \lb \alpha \in X(S) : d_\alpha > 0 \rb
\]
The \tbf{Weyl group} of $(G,S)$ is $W = N_G(S) / Z_G(S) = N_G(S) / S$. Usually, the functorial perspective on $W$ is unnecessary and we just care about the group of $k$-points, $W(k) = N_{G(k)} (S(k)) / S(k)$. Actually, most of the time we only care about the groups of $k$-points of all of this stuff.
\end{definition}

\begin{definition}
Let $\Phi$ be a (possibly non-reduced) root system. For $\alpha, \beta \in \Phi$, let
\[
	(\alpha, \beta) = \lb i\alpha + j\beta \in \Phi : i,j \in \Z_{\ge 1} \rb
\]
\end{definition}

\begin{lemma}
\label{commutator}
Let $G, k\footnote{Probably need to assume $k$ is characteristic zero, or at least not characteristic 2 or 3.}, S, \Phi$ be as in Definition \ref{isotropic group}.  Let $\alpha, \beta \in \Phi$ so that $\gamma = \alpha + \beta \in \Phi$ and $\alpha, \beta$ are not proportional\footnote{A relative root system may not be reduced, and we do not want $\alpha = \beta$ even though $2\alpha$ may be a root.}. There exist
\begin{itemize}
	\item Maps $e_\alpha:V_\alpha(k) \to G(k)$ aned $e_\beta:V_\beta(k) \to G(k)$
	\item For each element of $(\alpha, \beta)$ a function $e_{i\alpha+j\beta}:k \to G(k)$
	\item For each element of $(\alpha, \beta)$ a function\footnote{$N_{ij}^{\alpha \beta}$ should be homogeneous of degree $i$ in the first input and homogeneous of degree $j$ in the second input, but I am not sure if this is actually needed in any way for the construction.} $N_{ij}^{\alpha \beta}:V_\alpha(k) \times V_\beta(k) \to V_{i\alpha+j\beta}(k)$
\end{itemize}
such that for all $u \in V_\alpha(k)$ and $v \in V_\beta(k)$ we have
\begin{align*}
	\Big[ e_\alpha(u), e_\beta(v) \Big] &= \prod_{(\alpha,\beta)} e_{i\alpha + j\beta} \Big( N_{ij}^{\alpha \beta}(u,v) \Big)
\end{align*}
\end{lemma}
\begin{proof}
\incomplete
\end{proof}

\begin{lemma}
Let $G, k, S, \Phi$ be as in Definition \ref{isotropic group}. Let $\alpha, \gamma \in \Phi$ be roots of the same length. There exist 
\begin{itemize}
	\item An element\footnote{The element $w_{\gamma \to \alpha}$ should be in $N_{G(k)}(S(k))$ and satisfy $w^2 = 1$, but I am not sure if this is necessary to the calculation.} $w_{\gamma \to \alpha} \in G(k)$
	\item A function\footnote{I am pretty sure $\varphi_{\gamma \to \alpha}$ should be an isomorphism of vector spaces.} $\varphi_{\gamma \to \alpha}:V_\gamma(k) \to V_\alpha(k)$
\end{itemize}
such that for all $u \in V_\alpha(k)$ we have
\begin{align*}
	w_{\gamma \to \alpha} \cdot e_\gamma(u) \cdot w_{\gamma \to \alpha} \inv = e_\alpha \Big( \varphi_{\gamma \to \alpha}(u) \Big)
\end{align*}
\end{lemma}
\begin{proof}
\incomplete
\end{proof}

\begin{lemma}
Let $G, k, S, \Phi$ be as in Definition \ref{isotropic group}. Let $\alpha, \beta \in \Phi$ so that $\gamma = \alpha + \beta \in \Phi$ and $\alpha, \beta$ are not proportional, and let $N_{ij}^{\alpha \beta}$ be the function from Lemma \ref{commutator}. For every $u \in V_\alpha(k)$ and $v \in V_\beta(k)$, there exists an element $h \in G(k)$ such that
\[
	h_\gamma(u,v) \cdot \Bigg( \prod_{(\alpha,\beta)} e_{i\alpha+j\beta} \big(N_{ij}^{\alpha \beta}(u,v) \big) \Bigg) \cdot h_\gamma(u,v) \inv = \incomplete
\]
\end{lemma}
\begin{proof}
I'm not even sure what the right hand side of the equation above is supposed to be  \incomplete
\end{proof}

\end{document}